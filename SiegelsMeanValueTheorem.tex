
%-------------------------------------------------------------------
% SiegelsMeanValueTheorem.tex
%-------------------------------------------------------------------

\documentclass[11pt]{article}

% Usepackages
%-------------------------------------------------------------------

\usepackage{amsmath}
\usepackage{amsthm}
\usepackage{amssymb}
\usepackage{amsfonts}
\usepackage{bbding}
\usepackage{bm}
\usepackage{caption}
\usepackage{contour}%for writing sc bf
\usepackage[dvipsnames]{xcolor}
\usepackage{esvect}
\usepackage{enumerate}
\usepackage{enumitem}
\usepackage{float}
\usepackage{graphicx}
%\usepackage[linktoc=all]{hyperref}
\usepackage{mathrsfs}
\usepackage{mathtools}
\usepackage[normalem]{ulem}%for writing a underlined text
\usepackage{physics}
\usepackage{relsize}
\usepackage{slantsc}
%\usepackage{titlesec}
\usepackage[top=3cm, bottom=2.5cm, left=2.5cm, right=2.5cm]{geometry}
\usepackage{thmtools}
\usepackage{tfrupee}
%\usepackage{titling}
\usepackage{tikz}

\graphicspath{{Images/}}

% Environments
%-------------------------------------------------------------------

\newtheorem{theorem}{Theorem}[section]
\newtheorem{lemma}[theorem]{Lemma}
\newtheorem{corollary}[theorem]{Corollary}
\newtheorem{example}{Example}[section]
\newtheorem{exercise}{Exercise}[section]
\newtheorem{remark}[theorem]{Remark}

% Additional Environments
\newtheorem{smvt}[theorem]{Siegel's Mean Value Theorem}
\newtheorem{gsmvt}[theorem]{General Siegel's Mean Value Theorem}
\newtheorem{smvtp}[theorem]{Mean Value Theorem for primitive matrices}

\theoremstyle{definition}
\newtheorem{definition}[theorem]{Definition}

%-------------------------------------------------------------------
% Unnumbered Environments
%-------------------------------------------------------------------

\newtheorem*{uremark}{Remark}

% Newcommands
%-------------------------------------------------------------------

\newcommand{\mychi}{\raisebox{2pt}{$\chi$}}

%-------------------------------------------------------------------
% Numbering of equations
%-------------------------------------------------------------------

% \numberwithin{equation}{section}

%-------------------------------------------------------------------
% QED symbol
%-------------------------------------------------------------------

\renewcommand{\qedsymbol}{$\blacksquare$}

\begin{document}

\title{\bfseries Siegel's Mean Value Theorem}

\author{Mahbub Alam}

\date{}

\maketitle

% Abstract
\begin{abstract}
    This article is an attempt to understand and rewrite the proof of Siegel's Mean Value Theorem in relatively modern language.
\end{abstract}

\section{Introduction}
In 1944, Siegel~\cite{siegel45} proved the following mean value theorem:
\[
    \int\limits_{\mathrm{SL}(n,\mathbb{R}) / \mathrm{SL}(n,\mathbb{Z})} \widehat{f}(\Lambda ) \, \mathrm{d} \mu (\Lambda ) = \int_{\mathbb{R}^n}f(x) \, \mathrm{d} x
\]
where $f\in L^1(\mathbb{R}^n )$, $\widehat{f}$ (called the Siegel transform of $f$) is defined as, $\widehat{f}( \Lambda ) = \displaystyle\sum_{0 \neq v \in \Lambda } f(v)$, $\Lambda$ is any unimodular lattice in $\mathbb{R}^{n}$, $\mathrm{SL}(n, \mathbb{R})/ \mathrm{SL}(n, \mathbb{Z})$ is the space of all unimodular lattices and $\mu$ is a probability measure on $\mathrm{SL}(n,\mathbb{R})/\mathrm{SL}(n,\mathbb{Z})$.

Siegel's paper was simplified by Macbeath and Rogers~\cite{macbeathrogers58} in a 1957 paper where they prove a general mean value theorem asserted by Siegel at the end of his paper.
We will discuss this mean value theorem and some of its implications together with an explicit volume calculation of $\mathrm{SL}(n,\mathbb{R})/\mathrm{SL}(n,\mathbb{Z})$, i.e., the Minkowski-Siegel formula.

\section{Revisiting Gauss Circle Problem}
Let, $B \coloneqq B(0, r) \subseteq \mathbb{R}^2$.
Consider the function $\mychi _B$.
Define,
\[
    {\mychi _B}_*(x) = \sum_{k \in \mathbb{Z}^2} \mychi _B(x-k)
\]
Then, ${\mychi _B}_*(x) \sim \pi r^2$ as $r \to \infty\, \forall x \in \mathbb{R}^2$.
Also we have the following mean value theorem,
% Align
\begin{align*}
    \int_{\mathbb{R}^2} \mychi _B (x)\, \mathrm{d} x &= \displaystyle\sum_{k \in \mathbb{Z}^2} \, \int\limits_{[0,1)^2 - k} \mychi _B (x) \, \mathrm{d} x\\
    &= \displaystyle\sum_{k \in \mathbb{Z}^2} \int_{[0,1)^2} \mychi _B (x-k) \, \mathrm{d} x\\
    &= \int_{[0,1)^2} {\mychi _B}_*(x) \, \mathrm{d} x
\end{align*}
i.e., the average of the number of points of $\mathbb{Z}^2$ in a ball of radius $r$ centered at $x \in [0,1)^2$ is the volume of the ball.

% Corollary
\begin{corollary}[Blichfeldt (1914), Steinhaus (1999)]\label{cor1}
    Let $K$ be a Lebesgue measurable set in $\mathbb{R}^2$ with $ \lambda (K) > 1$, where $\lambda$ is the Lebesgue measure on $\mathbb{R}^2$, then $\exists \, x_1 , x_2 \in K$ such that $x_1 - x_2 \in \mathbb{Z}^2$.
\end{corollary}

\section{The Minkowski-Siegel measure $\mu$ on $\mathrm{SL}(n,  \mathbb{R})$}
Let $E \subseteq \mathrm{SL}(n,  \mathbb{R})$ be measurable and $\alpha > 0$.
Define, $E(\alpha )= \{tg: t \in (0,\alpha ], g \in E\} \subseteq \mathrm{GL}^+(n, \mathbb{R}) \subseteq \mathbb{R}^{n^2}$ (the cone on $E$) and also define $ \mu (E) = \lambda (E(1))$, $\lambda$ being the Lebesgue measure on $\mathbb{R}^{n^2}$.

Let $g\in \mathrm{SL}(n,  \mathbb{R})$, then $\mu (Eg)= \lambda ((Eg)(1))=\lambda (E(1)g)=|g|^n \lambda (E(1))= \mu (E)$, where $|g|$ stands for the determinant of $g$.
So $\mu$ is right invariant.
Similarly it is left invariant.
Therefore, $\mu$ is a Haar measure on $\mathrm{SL}(n,\mathbb{R})$.

% Theorem
\begin{theorem}[Weil's integration formula]\label{thm}
    Let $G$ be a locally compact topological group and $H$ be a closed subgroup of $G$.
    Let $\Delta_G$ and $\Delta _H$ be the modular functions of $G$ and $H$ so that $\Delta _G\big|_H = \Delta _H$.
    Then $\exists$ a $G$-invariant Borel measure $\nu$ on $G/H$, which can be normalized so that for all integrable functions $f$ on $G$,
    \[
        \int_{G} f(g) \, \mathrm{d} g = \int_{G/H}\int_{H} f(gh) \, \mathrm{d} h \, \mathrm{d} \nu (\bar{g})
    \]

\end{theorem}

Therefore, from Weil's integration formula we see that $\mu$ gives rise to measure on $\mathrm{SL}(n,\mathbb{R})/\mathrm{SL}(n,\mathbb{Z})$, which we will also call $\mu$ as there will be no confusion.
It is also clear from Weil's integration formula that integration over $\mathrm{SL}(n,\mathbb{R})/\mathrm{SL}(n,\mathbb{Z})$ with respect to $\mu$ can viewed as integration over $F_n$ with respect to $\mu$, i.e., for $\rho $ integrable on $\mathrm{SL}(n,\mathbb{R})/\mathrm{SL}(n,\mathbb{Z})$
\[
    \int_{F_n} \rho (g) \, \mathrm{d} \mu (g) = \int\limits_{\mathrm{SL}(n,\mathbb{R})/\mathrm{SL}(n,\mathbb{Z})} \rho (\Lambda ) \, \mathrm{d} \mu (\Lambda )
\]

\section{Siegel's Mean Value Theorem}
We know that $\mathrm{SL}(n, \mathbb{Z})$ is a lattice in $\mathrm{SL}(n, \mathbb{R})$ and that $\mathrm{SL}(n, \mathbb{R})/\mathrm{SL}(n, \mathbb{Z})$ can be viewed as the space of all unimodular lattices in $\mathbb{R}^{n}$.
Let $F_n$ be a fundamental domain of $\mathrm{SL}(n,  \mathbb{Z})$ in $\mathrm{SL}(n,  \mathbb{R})$.
Then every unimodular lattice in $\mathbb{R}^{n}$ can be written as $g \mathbb{Z}^n$ for exactly one $g \in F_n$.

% Theorem
\begin{smvt}\label{smvt}
    If $f \in L^1 (\mathbb{R}^n)$ then,
    \[
        \int\limits_{\mathrm{SL}(n,  \mathbb{R}) / \mathrm{SL}(n,  \mathbb{Z})} \widehat{f}(\Lambda  ) \, \mathrm{d} \mu (\Lambda  ) = \mu \big(\mathrm{SL}(n,\mathbb{R})/\mathrm{SL}(n,\mathbb{Z})\big)\int_{\mathbb{R}^n}f(x) \, \mathrm{d} x
    \]
    where $\widehat{f}( \Lambda ) = \displaystyle\sum_{0 \neq v \in \Lambda } f(v)$, $\Lambda \in \mathrm{SL}(n, \mathbb{R})/\mathrm{SL}(n, \mathbb{Z})$.

    Equivalently, if $f \in L^1 (\mathbb{R}^n)$ then,
    \[
        \int_{F_n} \widehat{f}(g) \, \mathrm{d} \mu (g) = \mu (F_n)\int_{\mathbb{R}^n}f(x) \, \mathrm{d} x
    \]
    where $\widehat{f}(g) = \displaystyle\sum_{0 \neq v \in \mathbb{Z}^{n}} f(gv)$, $g \in F_n$.
\end{smvt}

If $K \subseteq \mathbb{R}^n $ is Borel measurable, then with $f=\mychi _K$ and $\widehat{\mychi _{K}}(g)= \# $ non-zero points of $ g \mathbb{Z}^n$ in $K$, the above says,
\[
    \frac{1}{\mu (F_n)}\int_{F_n} \widehat{\mychi _{K}}(g) \, \mathrm{d} \mu (g) = \lambda (K)
\]
This is why Siegel's Mean Value Theorem~\ref{smvt} is called a mean value theorem.

Let $1 \le \ell  \le n-1,\, \ell +m=n$ and $LI=$ Set of all $n \times \ell$ integer matrices with linearly independent columns.
At the end of~\cite{siegel45} Siegel claimed the following.

% Theorem
\begin{gsmvt}\label{gsmvt}
    If $f \in L^1 ( \mathbb{R}^{n \ell})$ then,
    \[
        \displaystyle \int_{F_n} \displaystyle\sum_{q \in LI} f(gq) \, \mathrm{d} \mu (g) = \mu (F_n) \int_{\mathbb{R}^{n \ell}} f(x) \, \mathrm{d} x
    \]
\end{gsmvt}

One might guess that taking $f= \mychi _K$ for $K \subseteq \mathbb{R}^{n \ell}$ Borel measurable, proof of the above theorem might reduce to calculating volume of certain set in two different ways.
Proceeding in this manner one can prove the slightly different theorem below, from which the above theorem can be deduced.

Define, $P = \{he: h \in \mathrm{SL}(n,  \mathbb{Z})\} \subseteq LI$ where, $e$= first $\ell$ columns of $I_n$ = $
\begin{pmatrix}
    I_ \ell \\
    0
\end{pmatrix}$.
That is, $P$ is the set of all $n \times \ell$ integral matrices that appear as the first $\ell$ columns of some matrix in $\mathrm{SL}(n,\mathbb{Z})$.

% Theorem
\begin{smvtp}\label{smvtp}
    If $f \in L^1(\mathbb{R}^{n\ell })$ then,
    \[
        \int_{F_n} \displaystyle\sum_{p \in P} f(gp) \, \mathrm{d} \mu (g)= \frac{m}{n} \mu (F_m) \int_{\mathbb{R}^{n \ell }}f(x) \, \mathrm{d} x
    \]
\end{smvtp}

With $\ell=1$ the above says,
% Equation
\begin{equation}\label{eq1.4}
    \int_{F_n} \displaystyle\sum_{\substack{v \in \mathbb{Z}^n \\ v \,\text{primitive}}} f(gv) \, \mathrm{d} \mu (g) = \frac{n-1}{n} \mu (F_{n-1}) \int_{\mathbb{R}^n} f(x) \, \mathrm{d} x
\end{equation}
\section{Proof of Siegel's Mean Value Theorem~\ref{smvt} assuming Mean Value Theorem for primitive matrices~\ref{smvtp}}\label{secn1.5}
% Beginning of the proof of Siegel's Mean Value Theorem
With $f=\mychi _K$ and $\widehat{\mychi _{K}}'(g)=\#$ primitive vectors of $g \mathbb{Z}^n$ in $K$, (\ref{eq1.4}) implies
% Equation
\begin{equation}\label{eq1.5}
    \int_{F_n}\widehat{\mychi _{K}}'(g) \, \mathrm{d} \mu (g) = \frac{n-1}{n} \mu (F_{n-1}) \lambda (K)
\end{equation}

For any  $x( \neq 0) \in g \mathbb{Z}^n \ \exists! \ \nu \in \mathbb{Z}_+$ such that $\frac{x}{ \nu }$ is primitive.
Therefore,
% Equation
\begin{equation}\label{eq1.6}
    \widehat{\mychi _{K}} (g) = \displaystyle\sum_{\nu \in \mathbb{Z}_+ } \widehat{\mychi _{\frac{1}{\nu }K}}'(g) \ \forall g \in F_n
\end{equation}
Note that $\lambda (\frac{1}{\nu }K) = \nu ^{-n} \lambda (K)$. Therefore, from (\ref{eq1.5}) and (\ref{eq1.6}) we get (using Lebesgue monotone convergence theorem),
% Equation
\begin{equation}\label{eq1.7}
    \int_{F_n} \widehat{\mychi _{K}}(g) \, \mathrm{d} \mu (g) = \frac{n-1}{n} \zeta (n) \mu (F_{n-1}) \lambda (K)
\end{equation}

\subsection{Minkowski-Siegel Formula}
Let $\rho >0$ and $C_ \rho = {[-\rho, \rho ]}^n$.
Then, $\lambda (C_ \rho )= \rho ^n \lambda (C_1)$.
We also have $\widehat{\mychi _{C_ \rho }}(g) \sim \rho ^n \lambda (C_1)$, i.e., $\rho ^{-n} \widehat{\mychi _{C_ \rho }}(g) \to \lambda (C_1)$ as $\rho \to \infty \ \forall g \in \mathrm{SL}(n, \mathbb{R})$.

From (\ref{eq1.7}) we have,
\begin{align}\label{eq1.8}
    \int_{F_n}\rho ^{-n} \widehat{\mychi _{C_ \rho }} (g)\, \mathrm{d} \mu (g) = \frac{n-1}{n} \zeta (n)\mu (F_{n-1}) \lambda (C_1)
\end{align}
With $\rho =1$ the above says,
% Equation
\begin{equation}\label{eq1.9}
    \int_{F_n} \widehat{\mychi _{C_1 }} (g)\, \mathrm{d} \mu (g)= \frac{n-1}{n} \zeta (n)\mu (F_{n-1})\lambda (C_1)
\end{equation}

We know from Minkowski's convex body theorem that $\widehat{\mychi _{C_1}}(g) \ge 2 \ \forall g \in F_n$.
Therefore, from (\ref{eq1.9}) we have a $L^1$ function on $F_n$ which is bounded away from zero and hence $F_n$ has finite measure.
% Lemma
\begin{lemma}\label{lem1.5.1}
    Let $Q$ be cube with side length unity and faces parallel to coordinate planes, then $\widehat{\mychi _{Q}}(g) \le \widehat{\mychi _{C_1}}(g) \, \forall g \in \mathrm{SL}(n, \mathbb{R})$.
\end{lemma}
\begin{proof}
    By Minkowski's Convex body theorem $\widehat{\mychi _{C_1}}(g) \ge 2$.
    Therefore, we need only to worry about the case when $\widehat{\mychi _{Q}}(g) \ge 3$.
    Let, $x_1, x_2, \ldots , x_N \,(N \ge 3)$ be the nonzero elements of $g\mathbb{Z}^n$ in $Q$, then
    \[	
        \pm(x_2-x_1), \ldots , \pm(x_N-x_1)
    \]
    are nonzero elements of $C_1 \cap g \mathbb{Z}^n$.
    One can choose $x_1$ in such a way that these elements are all distinct.
    Therefore,  $N \le 2(N-1) \le \widehat{\mychi _{C_1}}(g)$, as $N \ge 3$.
\end{proof}

% Lemma
\begin{lemma}\label{lem1.5.2}
    Let $\rho \ge 1$.
    Then, $ \rho ^{-n}\widehat{\mychi _{C_ \rho }}(g) \le 3^n \widehat{\mychi _{C_1}}(g) \, \forall g \in \mathrm{SL}(n, \mathbb{R})$.
\end{lemma}
\begin{proof}
    $C_ \rho $ can be contained inside ${([2 \rho ]+1)}^n$ cubes of side unity and faces parallel to coordinate planes, therefore from Lemma~\ref{lem1.5.1},
    \[	
        \frac{\widehat{\mychi _{C_ \rho }}(g)}{\widehat{\mychi _{C_1}}(g)} \le {([2 \rho ]+1)}^n \le {(2 \rho + \rho )}^n=3^n \rho ^n
    \]
\end{proof}

Therefore, taking limit as $\rho \to \infty$ in (\ref{eq1.8}), using dominated convergence theorem we get the recursion formula
% Equation
\begin{equation}\label{eq1.10}
    \mu (F_n)= \frac{n-1}{n} \zeta (n) \mu (F_{n-1})
\end{equation}
and hence using $ \mu (F_1) = 1$ we get the Minkowski-Siegel formula,
% Equation
\begin{equation*}
    \mu (F_n) = \frac{1}{n}\zeta (2) \zeta (3) \cdots \zeta (n)
\end{equation*}

\vspace{2em}
%  Finishing the proof of Siegel's Mean Value Theorem
From (\ref{eq1.7}) and (\ref{eq1.10}) we have,
\[
    \int_{F_n} \widehat{\mychi _{K}} (g) \, \mathrm{d} \mu (g) = \mu(F_n) \int_{\mathbb{R}^n } \mychi _K(x) \, \mathrm{d} x
\]
Hence, Siegel's Mean Value Theorem~\ref{smvt} is true for positive simple functions and therefore by taking limits it is true for positive functions and hence for all $L^1 $ functions $f$.
See~\cite{macbeathrogers58} for a proof of General Siegel's Mean Value Theorem~\ref{gsmvt}.

%  Minkowski-Hlawka Theorem
In 1944, E. Hlawka proved the following longstanding conjecture of Minkowski.
The following proof is due to Siegel~\cite{siegel45}.
% Corollary
\begin{corollary}[Minkowski-Hlawka Theorem]\label{cor2}
    If $B$ is a starshaped,  centrally symmetric body with volume less than $2 \zeta (n)$, then there exists a unimodular lattice $g \mathbb{Z}^n$ with no non-zero point of it inside $B$.
\end{corollary}
% Proof
\begin{proof}
    From (\ref{eq1.5}), after normalizing the measure on $F_n$, we get,
    % Equation
    \begin{equation}\label{eq1.11}
        \zeta (n) \int_{F_n}\widehat{\mychi _{B}}'(g) \, \mathrm{d} \mu (g ) = \lambda (B)
    \end{equation}
    If $B$ contains a nonzero point of $g\mathbb{Z}^n \ \forall g \in F_n$, then it contains at least two primitive vectors of $g\mathbb{Z}^n$ (since $B$ is starshaped and centrally symmetric), i.e., $\widehat{\mychi _{B}}'(g) \ge 2 \ \forall g \in F_n$.
    So,
    \[	
        \lambda (B) \ge \zeta(n) \int_{F_n}2 \, \mathrm{d} \mu (g) = 2 \zeta (n)
    \]
    Therefore, under the assumption of the theorem there is a positive measure set of unimodular lattices with no nonzero point in $B$.
\end{proof}

\section{Action of $\mathrm{SL}(n, \mathbb{Z})$ on $P$}
$\mathrm{SL}(n, \mathbb{Z})$ acts transitively on $P$ by left-multiplication.
The stabilizer of $e$ (= first $\ell$ columns of $I_n $) is, ${\mathrm{SL}(n, \mathbb{Z})}_e := \{h \in \mathrm{SL}(n, \mathbb{Z}): he = e\}$.
Then, $\mathrm{SL}(n, \mathbb{Z}) / {\mathrm{SL}(n, \mathbb{Z})}_e \cong P $.
For $p \in P$ define ${\mathrm{SL}(n, \mathbb{Z})}_p := \{h \in \mathrm{SL}(n, \mathbb{Z}): he = p\}$.

If $h \in {\mathrm{SL}(n, \mathbb{Z})}_e$, then $h=
\begin{pmatrix}
    I_\ell & \eta   \\
    0      & h'
\end{pmatrix}
$, where $\eta$, $h'$ are integer matrices with $|h'|=|h|=1$, i.e., $h' \in \mathrm{SL}(m,\mathbb{Z})$.
If $h_1 \in {\mathrm{SL}(n, \mathbb{Z})}_p$, then ${\mathrm{SL}(n,\mathbb{Z})}_p = h_1 {\mathrm{SL}(n, \mathbb{Z})}_e$.

\section{New coordinates on a open subset of $\mathrm{GL}^+(n, \mathbb{R})$}\label{secn1.6}
To prove Mean Value Theorem for primitive matrices~\ref{smvtp}, we want to show first that if $K \subseteq \mathbb{R}^{n \ell }$ is a Borel measurable set then,
\[
    \int_{F_n} \displaystyle\sum_{p \in P} \mychi _K (gp) \, \mathrm{d} \mu  (\bar{g}) = \frac{m}{n} \mu (F_m) \lambda (K)
\]
Which we will show by calculating the volume of a certain set in $\mathrm{SL}(n,\mathbb{R}	)$ in two different ways.
For that reason we need to change coordinates on a open subset of $  \mathrm{GL}^+(n, \mathbb{R})$.
% \vspace{2em}

Let $a \in  \mathrm{GL}^+(n, \mathbb{R}) $. Then,
% Align
\begin{align*}
    a &=
    \begin{pmatrix}
        x_1 & z_1  \\
        x_2 & z_2
    \end{pmatrix}
    =
    \begin{pmatrix}
        & z_1  \\
        \raisebox{.8em}{$x$} & z_2
    \end{pmatrix}
    \leftarrow
    \begin{pmatrix}
        & \ell \times m  \\
        \raisebox{1em}{$n \times \ell$} & m \times m
    \end{pmatrix}\\
    & \ \ \ \ \ \ \ \ \ \uparrow \\
    &
    \begin{pmatrix}
        \ell \times \ell & \ell \times m  \\
        m \times \ell    & m \times m
    \end{pmatrix}
\end{align*}
We will say that $a$ and $x =
\begin{pmatrix}
    x_1 \\
    x_2
\end{pmatrix} \in \mathbb{R}^{n\ell }$ are well-balanced if $|x_1| \neq 0$.
Let $W= \{a \in \mathrm{GL}^+(n, \mathbb{R}): a\text{ is well-balanced}\}$ and $W(\mathbb{R}^{n\ell }) = \{x \in \mathbb{R}^{n\ell } : x \text{ is well-balanced}\}$.
Then, $W$ and $W(\mathbb{R}^{n\ell })$ are open and full Lebesgue measure in their respective spaces.

For all $a \in W$, get $\widetilde{x}_1 \in \mathrm{GL}(m, \mathbb{R})$ so that $|x_1| \cdot |\widetilde{x}_1|=1$ and $\widetilde{x}_1$ depends smoothly on $x_1$.
For example, if $|x_1|= \pm \alpha ^m$, can choose
% Bmatrix
\[
    \widetilde{x}_1=
    \begin{pmatrix}
        \alpha ^{-1} &   \\
        \            & \alpha ^{-1} \\
        \            &              & \ddots \\
        \            &              &              & \pm \alpha ^{-1}
    \end{pmatrix}
\]

Consider the matrix equation,
% Equation
\begin{equation}\label{eq1.12}
    \begin{pmatrix}
        x_1 & z_1  \\
        x_2 & z_2
    \end{pmatrix}
    =a=
    \begin{pmatrix}
        x_1 & 0  \\
        x_2 & \widetilde{x}_1
    \end{pmatrix}
    \begin{pmatrix}
        I_\ell & y  \\
        0      & a'
    \end{pmatrix}
\end{equation}
with $y$ and $a'$ as variables.
As $\bigg|
\begin{pmatrix}
    x_1 & 0  \\
    x_2 & \widetilde{x}_1
\end{pmatrix} \bigg| =1$, it has unique smooth solution.
(See Subsection~\ref{1.9.2}).
This matrix equation is equivalent to the following equations,
% Equation
\begin{equation}\label{eq1.13}
    \begin{split}
        z_1 &= x_1y\\
        z_2 &= x_2y + \widetilde{x}_1 a'
    \end{split}
\end{equation}
Therefore, we can denote $a$ by $[x,y,  a']$.
Note that $|a|=|a'|$, i.e., $a' \in \mathrm{GL}^+(m, \mathbb{R})$.
For $X \subseteq W(\mathbb{R}^{n\ell })$, $Y \subseteq \mathbb{R}^{\ell m}$ and $A' \subseteq \mathrm{GL}^+(m, \mathbb{R})$ we define $[X,Y,A']=\{[x,y, a']: x \in X, y \in Y, a' \in A'\} \subseteq W$.
We also note that,
% Align
\begin{align}\label{eq1.14}
    \dfrac{\partial (x, z_1, z_2)}{\partial (x, y, a')} &= \frac{\partial (x)}{\partial (x)} \frac{\partial (z_1)}{\partial (y)} \frac{\partial (z_2)}{\partial (a')} = 1 \cdot |x_1|^m \cdot |\widetilde{x}_1|^m = 1
\end{align}

The set we are going to calculate volume of in two different ways is a set of the form $[X,Y,A']$.

\section{Measure of sets in $\mathrm{SL}(n,\mathbb{R})$}
% Lemma
\begin{lemma}\label{lem1.8.1}
    Let, $f \in L^1(0,1]$ and $G' \subseteq \mathrm{SL}(m, \mathbb{R})$, then
    \[	
        \int_{G'(1)} f(|a|) \, \mathrm{d} a = \mu (G') \int_{0}^{1} f( \tau ^{1/m}) \, \mathrm{d} \tau
    \]
\end{lemma}
\begin{proof}
    Suppose first that $f$ is the characteristic function of the interval $(0,\alpha )$ (or, $(0, \alpha ]$), where $\alpha \leq 1$.
    Then, $f(|a|)$ is the characteristic function of the set $|a| < \alpha $ (or, $|a| \leq \alpha $), so the left side is the measure of the set of points in $G'(1)$ such that $|a| < \alpha $ (or, $|a| \leq \alpha $), i.e., of $G'(\alpha ^{1/m})= \alpha ^{1/m}G'(1)$ with measure $\lambda(\alpha ^{1/m} G'(1))= \alpha ^m \mu (G')$.
    The right side is
    \[	
        \mu (G') \int_{0 < \tau ^{1/m} \le \alpha } \, \mathrm{d} \tau = \alpha ^m \mu (G')
    \]

    Suppose next that $f$ is the characteristic function of $(\alpha , \beta )$, where $0 < \alpha < \beta \le 1$.
    The result follows, since the characteristic function of $(\alpha , \beta )$ is the difference of the characteristic functions of $(0, \beta )$ and $(0, \alpha ]$.

    Therefore, the result is true for all positive simple functions and by taking limits the general result follows.
\end{proof}

% Lemma
\begin{lemma}\label{lem1.8.2}
    Let $G=[X,Y,G']$, where $X, Y$ are as before and $G' \subseteq \mathrm{SL}(m, \mathbb{R})$, then
    \[	
        \mu (G)= \frac{m}{n} \lambda (X) \lambda (Y) \mu (G')
    \]
\end{lemma}
\begin{proof}
    Let $g\in G$, then $g=[x,y,g']$.
    Then, $tg= [tx, y, t^{n/m}g']$ from (\ref{eq1.13}).
    Therefore, $G(1)$ consists of all $[x,y, a'] \in \mathrm{GL}^+(n, \mathbb{R})$, where $x \in tX,\, y\in Y,\, a'= t^{n/m}g' \,(g' \in G'),\, 0 < t \le 1$.
    The range of the values of $a'$ is therefore $G'(1)$ and $t$ is a function of $a'$ given as,
    \[	
        |a'|= |t^{n/m}g'|=t^n |g'|=t^n
    \]
    So, using (\ref{eq1.14})
    \begin{align*}
        \mu (G)&= \int_{G(1)} \, \mathrm{d} x \, \mathrm{d} z_1 \, \mathrm{d} z_2\\
        &=\int\limits_{\substack{a' = t^{n/m}g', \\ g' \in G', \, 0 < t \le 1}} \mathrm{d} a' \int_{tX} \mathrm{d} x \int_{Y} \mathrm{d} y
    \end{align*}
    We have, $\displaystyle\int_{tX} \mathrm{d} x = \lambda (tX) = t^{n \ell} \lambda (X)= |a'|^ \ell \lambda (X)$, therefore using Lemma~\ref{lem1.8.1} we see that,
    \begin{align*}
        \mu (G)&= \lambda (X) \lambda (Y) \int_{G'(1)} |a'|^ \ell \, \mathrm{d} a'\\
        &= \lambda (X) \lambda (Y) \mu (G') \int_{0}^1 \tau ^{\ell/m} \, \mathrm{d} \tau \hfill\\
        &= \frac{m}{n} \lambda (X) \lambda (Y) \mu (G')
    \end{align*}
\end{proof}

\section{Proof of Mean Value Theorem for primitive matrices~\ref{smvtp}}\label{sec1.8}
% Beginning of the proof 
Let $K \subseteq W(\mathbb{R}^{n\ell })$ be Lebesgue measurable, $C= [0, 1)^{\ell m}$ be the unit cube in $\mathbb{R}^{\ell m}$.
Also let $K'= [K,C,F_m] \subseteq \mathrm{SL}(n, \mathbb{R})$.
On one hand we know from Lemma~\ref{lem1.8.2} that,
% Equation
\begin{equation}\label{eq1.15}
    \mu (K') = \frac{m}{n} \mu (F_m ) \lambda (C) \lambda (K)= \frac{m}{n}\mu (F_m)\lambda (K)
\end{equation}
On the other hand, using right invariance of the Haar measure $\mu$,
% Align
\begin{align}\label{eq1.16}
    \mu (K')&= \int_{\mathrm{SL}(n, \mathbb{R})}\mychi _{K'}(g) \, \mathrm{d} \mu (g) \nonumber \\
    &= \int_{F_n} \sum_{h \in \mathrm{SL}(n, \mathbb{Z})} \mychi _{K'}(g) \, \mathrm{d} \mu (g) \nonumber\\
    &= \int_{F_n} \sum_{p \in P} \sum_{h \in {\mathrm{SL}(n, \mathbb{Z})}_p} \mychi _{K'}(gh) \, \mathrm{d} \mu (g)
\end{align}

% Lemma
\begin{lemma}\label{lem1.8.3}
    (i) $\displaystyle\sum_{h \in {\mathrm{SL}(n, \mathbb{Z})}_e} \mychi _{K'}(gh) = \mychi _K(ge) \,\forall g \in \mathrm{SL}(n, \mathbb{R})$

    (ii) Let $p \in P$, then $\displaystyle\sum_{h \in {\mathrm{SL}(n, \mathbb{Z})}_p} \mychi _{K'}(gh) = \mychi _K(gp) \, \forall g \in \mathrm{SL}(n, \mathbb{R})$
\end{lemma}
\begin{proof}
    (i) If $g$ isn't well-balanced, then so isn't $ge$ and $gh \ \forall h \in {\mathrm{SL}(n, \mathbb{Z})}_e$, hence the lemma follows trivially.

    If $g$ is well-balanced, let $g= [x,y,g'] =
    \begin{pmatrix}
        x_1 & 0  \\
        x_2 & \widetilde{x}_1
    \end{pmatrix}
    \begin{pmatrix}
        I_\ell & y  \\
        0      & g'
    \end{pmatrix}
    $, also let that $h=
    \begin{pmatrix}
        I_ \ell & \eta   \\
        0       & h'
    \end{pmatrix}
    $.
    Here, $g' \in \mathrm{SL}(m, \mathbb{R}), h' \in \mathrm{SL}(m, \mathbb{Z})$.

    So, $gh=
    \begin{pmatrix}
        x_1 & 0  \\
        x_2 & \widetilde{x}_1
    \end{pmatrix}
    \begin{pmatrix}
        I_ \ell  & \eta + yh'  \\
        0        & g'h'
    \end{pmatrix}
    = [x, \eta + yh', g'h']
    $.
    Therefore,
    % Equation
    \begin{equation}\label{eq1.17}
        \displaystyle\sum_{h \in {\mathrm{SL}(n, \mathbb{Z})}_e} \mychi _{K'} (gh) = \displaystyle\sum_{\eta , h'} \mychi _K(x) \mychi _C(\eta +yh')\mychi _{F_m}(g'h')
    \end{equation}
    Note that $\exists!\, h'_0 \in \mathrm{SL}(m, \mathbb{Z})$ such that $g'h'_0 \in F_m$, also $\exists!\, \eta _0$ such that $\eta _0 + yh'_0 \in C$.
    Therefore, ignoring the terms which are zero on the right side of (\ref{eq1.17}), we have
    \begin{align*}
        \displaystyle\sum_{h \in {\mathrm{SL}(n, \mathbb{Z})}_e} \mychi _{K'} (gh) &= \mychi_K (x) \mychi _C(\eta _0 + yh'_0) \mychi _{F_m} (g' h'_0) \\
        &= \mychi _K(x)\\
        &= \mychi _K(ge)
    \end{align*}

    (ii) Fix $h_1 \in {\mathrm{SL}(n, \mathbb{Z})}_p$.
    Then all $h_2 \in {\mathrm{SL}(n, \mathbb{Z})}_p$ can be uniquely expressed as $h_1h$ with $h \in {\mathrm{SL}(n, \mathbb{Z})}_e$.
    So,
    \begin{align*}
        \displaystyle\sum_{h \in {\mathrm{SL}(n, \mathbb{Z})}_p} \mychi _{K'} (gh) &= \displaystyle\sum_{h \in {\mathrm{SL}(n, \mathbb{Z})}_e} \mychi _{K'}(gh_1h)\\
        &= \mychi _K (ghe)\\
        &= \mychi _K (gp)
    \end{align*}

\end{proof}

% Finishing the proof
Therefore, using Lemma~\ref{lem1.8.3} from (\ref{eq1.15}) and (\ref{eq1.16}) we get, if $K \subseteq W(\mathbb{R}^{n \ell})$ then,
\[
    \int_{F_n} \displaystyle\sum_{p \in P} \mychi_{K} (gp) \, \mathrm{d} \mu (g) = \frac{m}{n} \mu (F_m) \lambda (K)
\]

Let $K \subseteq \mathbb{R}^{n \ell }$ is arbitrary and $K_0 = K \cap W(\mathbb{R}^{n\ell })$.
Then, $\lambda (K)= \lambda (K_0)$.

For $p \in P$, let  $E_p=\{g \in F_n : gp \in K-K_0\}$, and $\mathcal{E}_p= \{g \in \mathrm{M}(n, \mathbb{R}): gp \text{ isn't well-balanced}\}$.
Then $\lambda ( \mathcal{E}_p)=0$, since $\mathcal{E}_p$ is the zero set of a polynomial.
Also $\mathcal{E}_p(1) =\mathcal{E}_p$ and $E_p \subseteq \mathcal{E}_p$.
Hence, $\mu (E_p) = 0$.
Therefore, $\Big\{g \in F_n : \displaystyle\sum_{p \in P} \mychi _K(gp) \neq \displaystyle\sum_{p \in P}\mychi _{K_0} (gp)\Big\} = \bigcup _{p \in P}E_p$ is a set of $\mu$-measure zero (as $P$ is countable).
Hence,
% Align
\begin{align*}
    \int_{F_n} \displaystyle\sum_{p \in P} \mychi _K (gp) \, \mathrm{d} \mu (g)
    &=\int_{F_n} \displaystyle\sum_{p \in P} \mychi _{K_0} (gp) \, \mathrm{d} \mu (g) \\
    &= \frac{m}{n} \mu (F_m) \lambda (K_0) \\
    &= \frac{m}{n} \mu (F_m) \lambda (K)
\end{align*}
So, we have proved Mean Value Theorem for primitive matrices~\ref{smvtp} for $f= \mychi _K, K \subseteq \mathbb{R}^{n \ell }$ measurable.
Now, considering positive simple functions and then taking limits of them we get Mean Value Theorem for primitive matrices~\ref{smvtp} for all functions.

\section{Remarks}
\subsection{Macbeath and Rogers' proof of $\mathrm{SL}(n, \mathbb{Z})$ being a lattice in $\mathrm{SL}(n, \mathbb{R})$}
Let $G$ be a locally compact topological group with a Haar measure $\mu$ and $\Gamma$ be a subgroup of $G$.
A Borel set $F \subseteq G$ is called a fundamental domain for $\Gamma$ if,
% Align
\begin{align*}\label{eq1.18}
    (i)& \ F \gamma _1 \cap F \gamma _2 = \varnothing, \ \text{if} \ \gamma _1 \neq \gamma _2 \in \Gamma \\
    (ii)& \ \bigcup_{\gamma \in \Gamma } F \gamma = G \\
\end{align*}
$\Gamma$ is called a lattice in $G$ if $\Gamma$ is discrete and there is a finite volume fundamental domain for $\Gamma$.

In~\cite{macbeathrogers58}, Macbeath and Rogers first constructed a fundamental domain for $\mathrm{SL}(n,\mathbb{Z})$ in $\mathrm{SL}(n,\mathbb{R})$ (which they call $F_n$).
Note that the proof of Mean Value Theorem for primitive matrices~\ref{smvtp} (Section~\ref{sec1.8}) do not require the assumption that $ \mu (F_n) < \infty$.
Then giving the same arguments as in Section~\ref{secn1.5}, they also showed that $\mathrm{SL}(n,\mathbb{Z})$ is a lattice in $\mathrm{SL}(n,\mathbb{R})$.

\subsection{Remarks on Section~\ref{secn1.6}}\label{1.9.2}
We want to view the first $\ell$ columns of $\mathrm{SL}(n,\mathbb{R})$ and $\mathrm{GL}^+(n,\mathbb{R})$ matrices as $\mathbb{R}^{n \ell }$.
But all elements in $\mathbb{R}^{n \ell }$ cannot occur as the first $\ell$ columns of some matrix in $\mathrm{GL}^+(n,\mathbb{R})$ (as those $\ell$ column vectors can be linearly dependent).

Let $P(\mathbb{R}^{n\ell }) = \{ae : a \in \mathrm{GL}^+(n, \mathbb{R})\} \subseteq \mathbb{R}^{n \ell }$.
Then, $W(\mathbb{R}^{n\ell}) \subseteq P(\mathbb{R}^{n\ell})$ and $\mathrm{GL}^+(n,\mathbb{R})$ acts transitively on $P(\mathbb{R}^{n\ell })$ by left-multiplication.
The stabilizer of $e$ is
\[
    {\mathrm{GL}^+(n, \mathbb{R})}_e = \{a \in \mathrm{GL}^+(n, \mathbb{R}) : ae = e\} = \bigg\{
    \begin{pmatrix}
        I_ \ell  & y  \\
        0        & a'
    \end{pmatrix}
    : y \in \mathbb{R}^{ \ell m}, a' \in \mathrm{GL}^+(m, \mathbb{R})\bigg\}
\]
Then $\mathrm{GL}^+(n, \mathbb{R})/{\mathrm{GL}^+(n, \mathbb{R})}_e \cong P(\mathbb{R}^{n\ell })$.
For $x \in P(\mathbb{R}^{n\ell })$ and $a_x \in {\mathrm{GL}^+(n, \mathbb{R})}_x := \{a \in \mathrm{GL}^+(n, \mathbb{R}) : ae = x\}$, we have ${\mathrm{GL}^+(n, \mathbb{R})}_x = a_x {\mathrm{GL}^+(n, \mathbb{R})}_e$.
For $x \in W(\mathbb{R}^{n\ell })$ we have a smooth choice $a_x =
\begin{pmatrix}
    x_1 & 0	\\
    x_2 & \widetilde{x}_1
\end{pmatrix}
\in {\mathrm{GL}^+(n, \mathbb{R})}_x$, as shown in Section~\ref{secn1.6}.


% Bibliography
\bibliographystyle{alpha}
\bibliography{bib}

\end{document}

%-------------------------------------------------------------------
% End of SiegelsMeanValueTheorem.tex
%-------------------------------------------------------------------

%-------------------------------------------------------------------
% Comments on the document
%-------------------------------------------------------------------

% Section 5 and 8 are folded differently.

